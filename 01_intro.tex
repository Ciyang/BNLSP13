Human language communication is efficient, in that people need not
always say every detail in order to be understood. Rather, speakers
often only say what is most relevant, and listeners can often
seemingly effortlessly grasp the intended meaning beyond what is
literally said. This ability to do pragmatic inference has been long
studied and the \emph{conversational implicature} theory by Grice
\cite{Grice} is one of the most prominent theories in the
field. However, since it is hard to formalize the \emph{Cooperative
  Principle} and the \emph{Conversational Maxims}, Grice's theory is
not precise enough to give empirically testable predictions,
especially quantitative predictions.

The Bayesian \emph{Rational Speech Act} (RSA) model attempts to
address this issue by using information-theoretic notions together
with the Bayesian inference framework
\cite{Frank2009,Frank,Bergen2012,GoodmanStuhlmuller2013:Knowledge-and-I}. It
has been shown that the model could yield quantitative predictions
that are highly correlated to actual human judgments
\cite{Frank,GoodmanStuhlmuller2013:Knowledge-and-I}.

Despite the RSA model's theoretical promise and empirical success, if
we analyze the design of the model, we will find several choices that
are not \emph{prima facie} obvious and might thus need further
consideration and justification. These choices have their alternatives
in closely related game-theoretic approaches to pragmatics
(e.g. \cite{Benz2007,Jager2013}), which have similar but slightly
different conceptual motivations. Hence it is important to
systematically compare the original RSA model with all these
alternatives and test their predictions on empirical data, so as to
gain a better understanding of the relation between these
models. Doing so will help illuminate the theoretical commitments and
empirical implications of each design choice, which will enhance our
understanding of the nature of human pragmatic inference.

The outline of the paper is as
follows. Section~\ref{sec:refer-games-rsa} introduce the referential
communication games investigated by \cite{Frank} and the original RSA
model proposed there. Section~\ref{sec:alternatives} analyzes a few
design choices in this model and introduces their alternatives in game
theoretic pragmatics, emphasizing the distinction between
\emph{belief}, \emph{goal} and \emph{action} and unifying various
models as different interpretations of these three
components. Section~\ref{sec:experiment} reports on the results of an
experiment similar to the one of \cite{Frank}. In
Section~\ref{sec:model-comparison}, we compare the predictions of
different models of our new data. Finally, we discuss the implications
of the results of the model comparison in
Section~\ref{sec:conclusion}, concluding that the original RSA model
can be improved on, in particular by taking speaker preferences into
account, but we also point out that a more complex picture of
probabilistic pragmatic reasoning about referential expressions awaits
further exploration.

%%% Local Variables: 
%%% mode: latex
%%% TeX-master: "main"
%%% TeX-PDF-mode: t
%%% End: