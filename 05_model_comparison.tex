Given the empirical data as described in the previous section, how can we decide which model is the best? One major challenge in model comparison is the fact that all models under investigation have some parameters that are unspecified. These include the speaker's degree of rationality $\lambda_\mathrm{S}$, the cost of adjectives $c$ for those speaker models that incorporate the speaker's preference, and the listener's degree of rationality $\lambda_\mathrm{L}$ for those listener models that have an action-oriented communication goal. Since there is no principled theory to determine the value of the parameters and no obvious way to directly measure them empirically (unlike the perceptual salience), we will only set some qualitatitive constraints on them from some conceptual consideration and remain uncertain about the exact values.   

\emph{Bayesian Model Averaging} provides us with a way to compare models in spite of the uncertainty about the exact specifications. The crucial idea is that we can use the \emph{hyperprior} to quantify our uncertainty about the exact values of the parameters in the model and calculate the average performance of the model $M$ in terms of the evidence, i.e. the probability to observe the data $D$ given that it is generated by the model: 
\begin{equation}\label{BMA}
\mathrm{Ev}(D \mid \mathcal{M})= \int \Pr(\theta) \cdot p(D | M, \theta)\, \mathrm{d}\theta,
\end{equation}
where $\mathcal{M}$ is the average model of the original model of interest with respect to the hyperprior $\Pr(\theta)$ that encodes our uncertainty about the parameter $\theta$. In this way we are able to compare two models when we are uncertain about their exact specifications. Note that we can also compare the same model with different hyperpriors, to draw conclusions about the properties of the model. For instance, for a speaker model $\sigma_{xy}$, if we want to know whether the data suggests a preference for nouns over adjectives, or the other way around, we can use hyperpriors that range over positive or negative costs, respectively, and compare the evidences of the resulting average models to see whether one is significantly better than the other. 

First we compare the speaker models $\sigma_{xy},\ x\in\{a,b\},y\in\{\mathcal{U},\mathcal{S}\}$. Each model has two parameters $\lambda_\mathrm{S}$ and $c$. We assume they are independent of each other:
\begin{equation}\label{hyper-speaker-independent}
\Pr(\lambda_\mathrm{S},c)=\Pr(\lambda_\mathrm{S}) \cdot \Pr(c) \enspace .
\end{equation}
We are uncertain about the rationality of the speaker:
\begin{equation}\label{hyper-speaker-lambda}
\Pr(\lambda_\mathrm{S})= \mathcal{U}_{(0,11)}(\lambda_\mathrm{S}),
\end{equation}
which is a uniform distribution over $(0,11)$. 

Since we are also interested in knowing whether the speaker has a preference in different utterances and in what direction if he does, we consider four types of hyperpriors for the cost $c$. The first hyperprior has $c=0$ with probability $1$, which means there is no speaker preference. The second hyperprior is $\mathcal{U}_{(0,0.4)}$, which means there is a preference for nouns and the third hyperprior is $\mathcal{U}_{(-0.4,0)}$ which means the preference is for adjectives. Finally, the hyperprior $\mathcal{U}_{(-0.4,0.4)}$ means that the preference exists, without commitment to either direction.

The comparison result of the speaker models are shown in Table \ref{table:speaker mod}. We can see that the data very strongly supports speaker models that don't take listener's perceptual salience into account. It also very strongly supports speaker models that incorporate the preference in utterances, even though the evidence is not enough to determine the direction of the preference. Finally, there is a consistent pattern that action-oriented models are slightly better than their belief-oriented counterparts, even though the differences are in general not significant.  

\begin{table}[htb] 
\caption{Speaker Model Comparison}
  \centering 
  \begin{tabular}{cccc}
    rank & model & preference & evidence \\
    \hline 
    1 
    & $\sigma_{\mathrm{a}\mathcal{U}}$ 
    & nouns
    & 1.01E-014
    \\
    2 
    & $\sigma_{\mathrm{b}\mathcal{U}}$ 
    & nouns
    & 7.33E-015
    \\
    3 
    & $\sigma_{\mathrm{a}\mathcal{U}}$ 
    & adj.~or nouns \quad
    &  5.04E-015
    \\
    4 
    & $\sigma_{\mathrm{b}\mathcal{U}}$ 
    &  adj.~or nouns
    & 4.02E-015
    \\     \\
    5 
    & $\sigma_{\mathrm{a}\mathcal{U}}$ 
    &  none
    & 4.93E-030
    \\
    6 
    & $\sigma_{\mathrm{b}\mathcal{U}}$ 
    &  none
    & 3.56E-030
    \\
    $\vdots$
    \\
    9 
    & $\sigma_{\mathrm{a}\mathcal{S}}$ 
    & nouns
    & 8.33E-036
    \\
    $\vdots$
    \\
    13 
    & $\sigma_{\mathrm{b}\mathcal{S}}$ 
    & nouns
    & 5.66E-061
    \\
    $\vdots$
  \end{tabular}
 
  \label{table:speaker mod}
\end{table}

This result suggests that the speaker does not take into account the preceptual salience of the listener in production, while having her own preference in different utterances. 


Now we compare the listener models, each of which has a speaker model nested inside as the belief of the listener. Since we want to know how such a belief correlates with the actual production, we put two kinds of hyperpriors under investigation. On the one hand, the flat-independent hyperpriors treat different parameters as  independent:
\begin{equation}\label{hyper-listener-independent}
\Pr(\lambda_\mathrm{S},c,\lambda_\mathrm{L})=\mathcal{U}_{(0,11)}(\lambda_\mathrm{S}) \cdot \Pr(c) \cdot  \mathcal{U}_{(0,11)}( \lambda_\mathrm{L}) \enspace .
\end{equation}
On the other hand, the true-correlated hyperpriors use the data from the speaker condition to estimate the hyperpriors for $\lambda_\mathrm{S}$ and $c$, and let $\lambda_\mathrm{L}$ have the same distribution as $\lambda_\mathrm{S}$:
\begin{equation}\label{hyper-listener-independent}
\Pr(\lambda_\mathrm{S},c,\lambda_\mathrm{L})=\mathcal{U}_{(0,11)}(\lambda_\mathrm{S} \mid D_\mathrm{S}, M_\mathrm{S}) \,\Pr(c\mid D_\mathrm{S}, M_\mathrm{S})  \,\mathcal{U}_{(0,11)}( \lambda_\mathrm{L} \mid D_\mathrm{S}, M_\mathrm{S}),
\end{equation}
and thus put a stronger constraint on the listener's belief about the speaker. 

The comparison result of the speaker models are shown in Table \ref{table:listener mod}. The best model, which is substantially better than the rest, is action-oriented and takes perceptual salience into account. It has a true-correlated speaker model which is action-oriented and does not take perceptual salience nor speaker preference into account. Note that the nested speaker model is different from the best actual speaker model only in that it does not incorporate speaker preference.  

\begin{table}[htb] 
\caption{Listener Model Comparison}
  \centering 
  \begin{tabular}{ccccc}
    rank
    & model
    & hyperprior
    & sp.~pref.
    & evidence 
    \\ \hline
    1 
    & $\rho_{\mathrm{a}\mathcal{S}}(\sigma_{\mathrm{a}\mathcal{U}})$
    & true correlated
    & $-$
    & 1.11E-003
    \\ \\
    2 
    & $\rho_{\mathrm{ a}\mathcal{S}}(\sigma_{\mathrm{b}\mathcal{U}})$
    & true correlated
    & $-$
    & 2.36E-004
    \\     
    3 
    & $\rho_{\mathrm{b}\mathcal{U}}(\sigma_{\mathrm{b}\mathcal{S}})$
    & flat independent \quad
    & adj.~or nouns \quad
    & 1.29E-004
    \\     
    4 
    & $\rho_{\mathrm{a}\mathcal{U}}(\sigma_{\mathrm{b}\mathcal{S}})$
    & flat independent
    & none
    & 9.94E-005
    \\     
    5 
    & $\rho_{\mathrm{a}\mathcal{U}}(\sigma_{\mathrm{b}\mathcal{S}})$
    & flat independent
    & adj.~or nouns
    & 8.79E-005
    \\
    $\vdots$\\
    14 
    & $\rho_{\mathrm{b}\mathcal{S}}(\sigma_{\mathrm{b}\mathcal{U}})$
    & flat independent
    & adj.~or nouns
    & 5.31E-005
    \\
    $\vdots$\\
    19 
    & $\rho_{\mathrm{b}\mathcal{S}}(\sigma_{\mathrm{b}\mathcal{U}})$
    & true correlated
    & $-$
    & 2.91E-005
    \\
    $\vdots$
  \end{tabular}
  \label{table:listener mod}
\end{table}

Combining the model comparison results for both the speaker and the listener models, it seems that if we treat the speaker's preference as some lexical salience which together with the listener's perceptual salience constitutes the contextual salience of the referential game, then contextual salience actually manifests \emph{lack of} common knowledge between the speaker and the listener rather than the presence of it, since they both have some preference that biases their decisions but they are unaware of each other's such preference. However, there is indeed something common between the speaker and the listener, i.e. the distribution of the degree of rationality. Also, the listener's belief about the speaker is correct modulo the speaker's private lexical salience. 
