Our main contributions are as follows: (1) We emphasized the distinction between belief, goal and action, systematically examined the RSA model, connected it to the game-theoretic approaches. (2) We tested the RSA model with a family of alternatives on the forced choice experiment and drew attention to the speaker's preference. Our data seems to suggest a different story about common knowledge --- what is commonly known is the Gricean pragmatic components, while both sides have his own private bias unknown to the other. (3) We explored the possibility of speakers doing second-order pragmatic inference and showed that it seems hard for them to do so. 

Our work shows that careful conceptual analysis of the design choices for quantitative models can lead to further improvement in predictive power. Of course, more data from a wide range of further experiments are needed for a full understanding of pragmatic inference about referential expressions.

