Our experiment and model comparison suggest the following: (1) On the conceptual level, 
it is helpful to clarify the distinction between belief, goal and action, and the Bayesian framework provides us with a natural way to structure these components into a formal model. By examining each component systematically, we can explicitly spell out the underlying assumptions and formulate alternative hypotheses to be tested. (2) On the empirical level, we tested the RSA model with a family of variants motivated from the game-theoretic pragmatic perspectives on communication, and in particular, we investigated the speaker's preference as well as the listener's perceptual salience. Our data showed a more intricate picture about the roles various factors play in pragmatic reasoning about referential expressions. There is some correlation between the speaker's and the listener's rationality,  while either side has his own bias and is unaware of the other's. Also, an action-based view might better reflect the goal of communication, at least in the forced-choice experimental setting. Understanding the relation between the forced-choice and the betting paradigms would be an important next step.

Our work shows that careful conceptual analysis of the design choices for quantitative models can lead to a better understanding of the phenomenon and further improvement in the formal model's predictive power. Of course, our preliminary experiment should only serve as a start and more empirical data are needed for a robust pragmatic theory of people's reasoning about referential expressions. 

%%% Local Variables: 
%%% mode: latex
%%% TeX-master: "main"
%%% TeX-PDF-mode: t
%%% End: