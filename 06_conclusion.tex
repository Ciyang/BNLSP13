Our experiment and model comparison suggest the following. (1) On the
conceptual level, it is helpful to clarify the distinction between
belief, goal and action, and the Bayesian framework provides us with a
natural way to structure these components into a formal model. By
examining each component systematically, we can explicitly spell out
the underlying assumptions and formulate alternative hypotheses to be
tested. (2) On the empirical level, we tested the RSA model with a
family of variants motivated from a game theoretic perspective on
communication, and in particular, we investigated the speaker's
preference as well as the listener's perceptual salience. Our data
showed a more intricate picture of the roles that various factors play
in pragmatic reasoning about referential expressions. We found
evidence for a correlation between the speaker's and the listener's
rationality. Either side appears to have his own biases, but appears
to be negligent of the other's. Also, an action-based view might
better reflect the goal of communication, at least in the
forced-choice experimental setting that we used. Understanding the
relation between the forced-choice and the betting paradigms would be
an important next step to gain more insight into the differences
between action- and belief-based goal structures.

The set of models we were able to compare explicitly here does clearly
not exhaust the space of plausible candidates. For example, we
excluded listener models in which the literal listener takes the
salience prior into account. As an anonymous reviewer points out, this
may seem paradoxical because the best listener models of our
comparison do assume that the actual pragmatic listener takes salience
into account, but does not believe that the speaker believes that he
does. But actually, there is no friction here at all, because the
listener model that is embedded in a speaker model is a different one
from the full pragmatic listener model. If we think of the pragmatic
listener as simulating his own hypothetical speaker behavior, there is
nothing puzzling about factoring in salience only at a higher level of
pragmatic inference.

A similar issue arises with respect to the rationality parameters in
action-based listener models. We assumed that there are two
rationality parameters, one for the listener's actual rationality and
one for the listener's beliefs about the speaker's rationality. But we
also saw that a model that correlates the set of joint values of these
parameter pairs gave the best predictions of our data. This raises
many interesting issues for further research, most of which we cannot
answer here. The most obvious idea is, of course, to equate the
speaker's and the listener's lambda and so that the listener would
believe that the speaker is exactly as rational as the listener
himself. Surprisingly, this model achieves a very poor fit on our
data, despite its parsimony. More research into correlations in
perspective-taking and beliefs in rationality will hopefully shed more
light on this fascinating issue.

In summary, our work shows that careful conceptual analysis of the
design choices for quantitative models can lead to a better
understanding of the phenomenon and further improvement in the formal
model's predictive power. Of course, our restricted empirical data can
only serve as a start and more data is needed to fuel further model
comparison towards a more robust pragmatic theory of people's
probabilistic reasoning about referential expressions.

%%% Local Variables: 
%%% mode: latex
%%% TeX-master: "main"
%%% TeX-PDF-mode: t
%%% End: