Our experiment and model comparison suggest the following: (1) on the conceptual level, 
it is helpful to clarify the distinction between belief, goal and action, and by systematically examining each component in the formal model we can more explicitly spell out the underlying assumptions and formulate alternatives to be tested. (2) on the empirical level, we tested the RSA model with a family of alternatives on a forced-choice experiment and investigated the speaker's preference as well as the listener's perceptual salience. Our data seem to suggest a different story about common knowledge --- what is commonly known is the Gricean pragmatic components, while both sides have his own private bias unknown to the other.

Our work shows that careful conceptual analysis of the design choices for quantitative models can lead to a better understanding of the phenomenon and further improvement in the formal model's predictive power. Of course, more empirical data are needed for a robust pragmatic theory of people's reasoning about referential expressions. 

